\documentclass[12pt]{article}
\usepackage{geometry} 
\geometry{a4paper}
\usepackage{listings}
\usepackage[utf8]{inputenc}
\usepackage{listings}
\usepackage{xcolor}
\usepackage{graphicx}
\usepackage{float}
\usepackage{caption}
\usepackage{subcaption}
\usepackage{amsmath}
\usepackage{algorithm}
\usepackage[noend]{algpseudocode}
\usepackage[shortlabels]{enumitem}
\usepackage[overload]{empheq}

\definecolor{codegreen}{rgb}{0,0.6,0}
\definecolor{codegray}{rgb}{0.5,0.5,0.5}
\definecolor{codepurple}{rgb}{0.58,0,0.82}
\definecolor{backcolour}{rgb}{0.95,0.95,0.92}
\lstdefinestyle{mystyle}{
	backgroundcolor=\color{backcolour},   
	commentstyle=\color{codegreen},
	keywordstyle=\color{magenta},
	numberstyle=\tiny\color{codegray},
	stringstyle=\color{codepurple},
	basicstyle=\ttfamily\footnotesize,
	breakatwhitespace=false,         
	breaklines=true,                 
	captionpos=b,                    
	keepspaces=true,                 
	numbers=left,                    
	numbersep=5pt,                  
	showspaces=false,                
	showstringspaces=false,
	showtabs=false,                  
	tabsize=2
}
\lstset{style=mystyle}

\title{Programming project}
\author{Umberto Cocca}
\date{} 

\begin{document}
\maketitle
\noindent \textbf{The Problem}\\
Let R be an axis-parallel rectangle and $D = \{d_1, ..., d_n\}$ a collection of closed circular disks.
Each disk $d_i$ is represented by its center $p_i$, which lies within $R$, and its positive radius $r_i$. Disks may extend outside of $R$, and one disk may be contained within another.

\begin{itemize}
	\item If every point of $R$ lies within at most one disk of $D$ then we say that the
	elements of $D$ form a packing of $R$ . Implement a plane-sweep $O(n log n)$ time algorithm
	that determines whether $D$ is a packing of $R$
	\item If every point of $R$ lies within at least one disk of $D$ then we say that the elements of $D$ form a cover of $R$. Implement a plane-sweep $O((n+m) log n)$ time algorithm that determines whether $D$ is a cover of $R$, where m is the number of intersection points lying within $R$ between the boundaries of disks. The running time should not depend on
	the number of intersection points that lie outside of $R$.
\end{itemize}


\noindent \textbf{Solution}\\
For the first problem if I proof that exists at least an intersection between two circles then the elements of $D$ do not form a packing of $R$. \\

\noindent For the second problem if I proof that exists at least one intersection h where h is not within any circule then we can state that the elements of $D$ do not form a cover of $R$. \\

\noindent \textbf{Implementation}\\
The programming language used is Python with the following main modules:
\begin{itemize}
	\item cv2: visual presentation to project the points in 2d image and look at what happens at each iteration
	\item numpy: used especially for some tasks during the animation
	\item pdb: it is debugger for python
	\item math: to execute specific calculations
\end{itemize}

\noindent \textbf{Data structures}\\

\noindent \textbf{Time complexity}\\

\noindent \textbf{Space complexity}\\

\noindent \textbf{How to execute the code}\\

\end{document}